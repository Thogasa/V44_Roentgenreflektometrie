\section{Auswertung}
\subsection{Detektor-scan}
Der Detektor-scan scan wird wie in der Durchführung beschrieben durchgeführt. An die Mess wurde eine Funktion der Form 
\begin{equation}
    I(\alpha)= I\idx{max}\cdot \e^{-(\frac{x-\mu}{\sigma})^2}
\end{equation}
angenähert. Die kurve is zu sehen in \ref{fig:decscan}. Die dabei resultierenden parameter sowie die Halbwertsbreite FWHM ist
\input{build/paramsdecScan.tex}
wobei die Halbwertsbreite mit 
\begin{equation}
    FWHM = \sqrt{8 \text{ln}(2)}(\sigma)
\end{equation}
bestimmt wurde.
\begin{figure}
    \centering
    \includegraphics[width = 7cm]{build/plotdecScan.pdf}
    \caption{Messdaten des Detektor-scans, daran eine Gausskurve angenähert.}
    \label{fig:decscan}
\end{figure}
\subsection{Z-scan}
\label{chap:zscan}
Die Messwerte des Z-scans sind in Abbildung \ref{fig:zscan} zu sehen. Die Strahlenbreite, welche abzulesen war, beträgt
\input{build/ergzScan.tex}
\begin{figure}
    \centering
    \includegraphics[width = 7cm]{build/plotzScan.pdf}
    \caption{Messdaten des Z-scans. Identifiezierte Strahlenbreite ist des weiteren eingetragen.}
    \label{fig:zscan}
\end{figure}
\subsection{Rocking-scan}
Zur bestimmung des Geometriefaktor wurde der Rocking-scan durchgeführt. In Abbildung \ref{fig:rocking} sind die Messwerte zu sehen.
\begin{figure}
    \centering
    \includegraphics[width = 7cm]{build/plotrockingScan.pdf}
    \caption{Messdaten des Rocking-scan sowie die daraus Identifiezierten Geometriewinkel.}
    \label{fig:rocking}
\end{figure}
Die enden des zu Geometriefaktor korrigierenden bereichs wurden aus Abbildung \ref{fig:rocking} abgelesen. Der Geometriewinkel wurde als
arithmetisches Mittel beider Winkel bestimmt. Der daraus resultierte Wert ist
\input{build/parrockingScan.tex}
Mit der Strahlenbreite aus Kapitel \ref{chap:zscan}, wurde der theoretische Geometriewinkel 
\begin{equation}
    \alpha\idx{theorie} = \text{arcsin}\left(\frac{d\idx{0}}{D}\right) = 1.777°,
\end{equation}
mit $D=20\,\si{\milli\meter}$. Auf grund der eindeutigen Abweichung, des theoretischen Wertes und dem experimentel bestimmten, wird im folgenden
der $\alpha\idx{theorie}$ für die Geometriefaktorkorrektur verwendet.
\subsection{Reflektivitäts-scan}
Bei der Reflektivitätsmessung wird die Schichtdicke $d\idx{0}$ sowie der kritische Winkel $\alpha\idx{c}$ bestimmt. Dazu werden die Messwerte
zunächst korrigiert mit den Messwerten eines analogen difusen Scans. Diese Messwerte sind in Abbildun \ref{fig:refl} zu sehen. 
\begin{figure}
    \centering
    \includegraphics[width = 7cm]{build/plotreflecScan.pdf}
    \caption{Messdaten des Reflektions-scan, mit und ohne Geometriefaktorkorrektur. Des weiteren ist der kritische Winkel eingetragen.}
    \label{fig:refl}
\end{figure}
Des weiteren sind die Messdaten, welche mit dem Geometriefaktor korrigiert wurden, in gleicher Abbildung zu sehen. Aus den gemessenen 
Intensitäten wurde die Reflektivität errechnet mittels
\begin{equation}
    R = \frac{5I}{I\idx{max}}
\end{equation}
bestimmt. Die minima wurden visuell bestimmt, und das arithmetische Mittel der differenzen wurde berechnet. Aus diesem Mittelwert wird die 
Schichtdicke mittels Formel \eqref{eq:schicht} bestimmt. Die daraus resultierten ergebnisse sind 
\input{build/ergreflecScan.tex}
Zur Bestimmung der Schichtdicke des Polysterols wurde die Wellenlänge der $K\idx{\alpha, 1} = 1.541\,\si{\angstrom}$ verwendet.
\subsection{Bestimmung des Dispersionsprofil mittels des Parratt-Algorithmus}
Auf den gleichen Daten wie im vorherigen Kapitel wird der Parratt-Algorithmus angewendet. Das wird durchgeführt, um die das Dispersionsprofil
der Probe zu bestimmen. Der Parratt-Algorithmus is für raue Oberflächen angepasst, und kann entsprechend auch ein Maß, wie rau die Oberfläche
ist, abschätzen. Es wurden zwei Schichten betrachten, eine Polysterolschicht auf einer Siliziumschicht. Die Luft ist als Vakuum angenähert.
\begin{figure}
    \centering
    \includegraphics[width = 7cm]{build/plotparratt.pdf}
    \caption{Messdaten des Reflektions-scan korrigiert Geometriefaktorkorrektur. Eine Parrattkurve wurde händisch an die Messwerte angenähert.}
    \label{fig:refl}
\end{figure}
Es wurden die Parameter der Parrattkurve händisch an die Messdaten angenähert. Die dabei gefunden Werte sind
\begin{equation}
    \begin{aligned}
        \delta\idx{2} &= 7.6\cdot 10^{6} \\
        \delta\idx{3} &= 8.077\cdot 10^{6} \\
        \sigma\idx{1} &= 2.688\cdot 10^{9} \\
        \sigma\idx{2} &= 4.829\cdot 10^{10} \\
        b\idx{1} &= 3.863\cdot 10^{7} \\
        b\idx{2} &= 1.483\cdot 10^{5} \\
        d\idx{2} &=  813.9\,\si{\angstrom}.
    \end{aligned}
\end{equation}
Aus den Parametern kann anschließend der kritische Winkel mit Formel \eqref{eq:kritisch} bestimmt. Die kritischen Winkel betragen
\begin{equation}
    \begin{aligned}
        \alpha\idx{c, exp, Polysterol} &= 0.0039\,° \\
        \alpha\idx{c, exp, Silizium} &= 0.0040\,°
    \end{aligned}
\end{equation}