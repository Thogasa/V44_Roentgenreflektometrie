\section{Diskussion}
Beim detektor-scan wurde die Parameter 
\input{build/paramsdecScan.tex}
ermittelt. Wiederrum beim Z-scan wurde eine Strahlenbreite von 
\input{build/ergzScan.tex} 
bestimmt. Da es für diese Werte keine Theoriewerte, zum vergleich gibt kann keine Aussage über die Güte der Messung getroffen werden.
Bei dem Rocking-scan wurde ein Geometriewinkel von $\alpha\idx{G} = 2.26\,°$ ermittelt. Verglichen mit dem Theoriewert $\alpha\idx{G, Theorie} = 1.777$
hat dieser Wert eine abweichung von ca $27.2\,\%$. Da diese Werte keine Unsicherheiten haben, ist eine eindeutige Aussage nicht möglich,
die hoche relative Abweichung deutet allerdings auf eine Fehlerhafte Justage.\\
Die messdaten der Reflektionmessung betohnt die bisherige Vermutung, da das erwartete Plateau, klein und undeutig ist. Vergleicht man 
die Ergebnisse der Schichtdicke, vom Reflektionscan und der Parrattkurve, 
\begin{equation}
\begin{aligned}
    d\idx{0} &= (959.70 \pm 0.71)\,\si{\angstrom} \\
    d\idx{0} &= 680.\,\si{\angstrom} \\
\end{aligned}
\end{equation}
ist offensichtilich, dass diese nicht mit einander Übereinstimmen. Die Abweichung beträgt $(29.91 \pm0.05)\,\%$. Dies ließe sich mit dem 
hohen Rauschen in den Oszillationen erklären. Dadurch, dass die Unsicherheit nur aus der Abweichung der Differenzen bestimmt wurde,
ist diese eindeutig eine Unterschätzung des realen Fehlers.
Die Parameter der Parrattkurve welche mit Literaturwerten \cite{V44} vergleichbar sind, sind die $\delta$ werte. Diese sowie die 
korrespondierenden Literaturwerte sind:
\begin{equation}
    \begin{aligned}
        \delta\idx{2, Parratt} &= 8.9\cdot 10^{6} \\
        \delta\idx{3, Parratt} &= 3.6\cdot 10^{6} \\
        \delta\idx{2, Theorie} &= 5\cdot 10^{6} \\
        \delta\idx{3, Theorie} &= 7.6\cdot 10^{6}. \\
    \end{aligned}
\end{equation}
Die Abweichungen hier sind $r\idx{2} = 78\,\%$ und $r\idx{3} = 111\,\%$. Hier ist aufzufallen, dass beide werte eine hohe Abweichung hat.
Aufgrund der händischen Annäherung an die Messwerte, fehlen jedoch die Ungenauigkeiten, was einen echten Vergleich unmöglich macht. 
Dennoch die große Abweichung von $\delta\idx{2}$ spricht erneut für eine Fehlerhafte Justage. Die Ergebnisse des Parrattalgorithmus, welche 
nicht mit Literaturwerten vergleichbar sind, sind 
\begin{equation}
    \begin{aligned}
        \sigma\idx{1} &= 4.7\cdot 10^{9} \\
        \sigma\idx{2} &= 5.8\cdot 10^{10} \\
        b\idx{1} &= 5.3\cdot 10^{8} \\
        b\idx{2} &= 2.8\cdot 10^{6} \\
        d\idx{2} &=  680\,\si{\angstrom}.
    \end{aligned}
\end{equation}